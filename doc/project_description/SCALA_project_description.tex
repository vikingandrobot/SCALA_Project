% -----------------------------------------------------------------------
% --- DOCUMENT ---
% -----------------------------------------------------------------------
\documentclass[11pt, a4paper, french, twoside]{article}

% -----------------------------------------------------------------------
% --- PACKAGE ---
% -----------------------------------------------------------------------
\usepackage[french]{babel}

% Font
\usepackage[utf8]{inputenc}
\usepackage[T1]{fontenc}

% Marge du document
\usepackage[top=3.5cm,
	bottom=3cm,
	left=2cm,
	right=2cm,
	footskip=1.5cm,
	headheight=1.5cm,
	headsep=0.9cm]{geometry}

% Gérer les positionnement d'images
\usepackage{float}

% Import de fichier externe
\usepackage{graphicx}

% Mise en forme des URL
\usepackage{url}

% Informations sur un document compilé en PDF et les liens externes / internes
\usepackage{hyperref}

% Pour les entêtes
\usepackage{fancyhdr}

% -----------------------------------------------------------------------
% --- INFORMATION SUR LE DOCUMENT
% -----------------------------------------------------------------------

% Information sur le document
\hypersetup{
	pdfauthor = {Kirushnapillai Sathiya, Monteverde Mathieu, Zucca Michela},                    % Auteurs
	pdftitle = {Project Cours Scala},                         % Titre du document
	pdfsubject = {Description du projet},                % Sujet
	pdfstartview={FitH}}                            % ajuste la page à la largueur de l'écran

% -----------------------------------------------------------------------
% --- EN-TETE ET PIED DE PAGE ---
% -----------------------------------------------------------------------
\pagestyle{fancy}
\fancyhf{} % Supprime les entetes et pieds de page existants

\fancyhead[LE,RO]{Description du projet\\}
\fancyhead[LO]{\includegraphics[width=4cm]{images/logo_heig.png}}
\fancyfoot[LE,RO]{\thepage{}}
\fancyfoot[RE]{Kirushnapillai Sathiya, Monteverde Mathieu, Zucca Michela}
\fancyfoot[LO]{Project Cours Scala}
\renewcommand{\footrulewidth}{1pt}


\title{Project Cours Scala \\ Description du projet}
\author{Kirushnapillai Sathiya, Monteverde Mathieu, Zucca Michela}
\date{2018}


\begin{document}
	\selectlanguage{french}
	\graphicspath{ {images/} }
	
	% Espacement entre les lignes
	\newcommand{\HRule}{\rule{\linewidth}{0.5mm}}
	
	% Page de garde
	\begin{titlepage}
    \begin{center}

	 \vspace{0.5cm}
     {\fontsize{1.5cm}{1.8cm} \bf Project Cours Scala}\par
     \vspace{0.5cm}
     {\fontsize{0.9cm}{1.3cm} \selectfont Description du projet}\par
     \vspace{3cm}
     \vfill
        
        % Author and supervisor
        \begin{minipage}{0.4\textwidth}
        	\begin{flushleft} \large
        		\textbf{Auteurs:}\\
        		\textsc{Kirushnapillai} Sathiya \\
        		\textsc{Monteverde} Mathieu \\
        		\textsc{Zucca} Michela
        	\end{flushleft}
        \end{minipage}
        \begin{minipage}{0.4\textwidth}
            \begin{flushright} \large
                \textbf{Professeur:} \\
                \textsc{Fatemi} Nastaran \\
                \textbf{Assistant:} \\
                \textsc{Santamaria} Miguel  
            \end{flushright}
        \end{minipage}
    
        \vfill
    \begin{minipage}{0.4\textwidth}
    	\begin{flushleft} \large
       		\includegraphics[width=5cm]{images/logo_heig.png}
        \end{flushleft}

	\end{minipage}
	\begin{minipage}{0.4\textwidth}
	    \begin{flushright}
			\includegraphics[width=5cm]{images/logo-hes-so.jpg}
		\end{flushright}
	\end{minipage}


        % Bottom of the page
        \today
        
    \end{center}
\end{titlepage}

	\newpage~
	
	% Empty page 
	\thispagestyle{empty}
	\newpage
	
	% Recommencer la numérotation des pages à "1"
	\setcounter{page}{1}
	\newpage
	
	% Espacement des paragraphes
	\setlength{\parskip}{0.2cm}
	
	\section{Contexte}
	\label{sec:contexte}
		Ce projet s'effectue dans le cadre du cours SCALA 2018. L'objectif est de réaliser une application Web en utilisant la technologie Scala Play pour le back-end et Slick pour la base de données. Le choix de la technologie front-end est libre.
		
	\section{Description}
	\label{sec:description}
	
		\subsection{Buts}
		\label{subsec:buts}
		L'idée de ce projet est de proposer une plateforme Web permettant de partager des événements avec ses utilisateurs. Un utilisateur inscrit au préalable pourra enregistrer des événements (par exemple un concert, une journée portes-ouvertes, un festival, etc.) en spécifiant des informations comme par exemple la localisation (coordonnées GPS ou adresse), une description, une ou plusieurs dates et des horaires.
		
		Un visiteur du site pourra ensuite rechercher les événements situés dans une certaines zone (par exemple dans un rayon de 10km autour de Lausanne) en spécifiant différents paramètres comme la date ou l'heure. 
		
		Le but est d'offrir un service indépendant de tout réseau social permettant à ses utilisateurs de trouver facilement des événements culturels ou festifs autour d'eux. 
		
	\section{Structure}
	\label{sec:structure}
		La figure \ref{fig:er} illustre le schéma entité-relation de la base de données du projet. L'application comporte des utilisateurs (User), des organisations (Organization) et des évenements (Event). Pour chaque entité, les opérations de base (CRUD) seront fournies : Afficher, créer, modifier et supprimer. Cependant, seuls les utilisateurs liés à une organisation peuvent créer, modifier et supprimer des événements.
		
		Pour cette première ébauche, chaque événement possède un ou plusieurs thèmes (Theme) qui permettront aux visiteurs de filtrer leurs recherches. Les thèmes permettent également aux utilisateurs inscrits d'être notifié d'un nouvel événement selon leurs préférences (Ce dernier point est pour l'instant optionnel).
		
		Et enfin, l'application permettra également aux utilisateurs inscrits de commenter un événement.
		
		\begin{figure}[h]
			\centering
			\includegraphics[width=0.8\linewidth]{images/project_ER.png}
			\caption{Schéma entité-relation de la base de données}
			\label{fig:er}
		\end{figure}
	
\end{document}
